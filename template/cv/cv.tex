% LaTeX file for resume 
% This file uses the resume document class (res.cls)

\documentclass{res} 
%\usepackage{helvetica} % uses helvetica postscript font (download helvetica.sty)
%\usepackage{newcent}   % uses new century schoolbook postscript font 
\newsectionwidth{1pt}  % So the text is not indented under section headings
\setlength{\textheight}{11in} % set text height big enough for box
\topmargin=-.5in       % to start box .5in from top of page
\oddsidemargin=-.6in   % to start box .5in from left of page
    
\begin{document}
 
%%%%%%%%%%%%%%%%%%%%%%%%%%%%%%%%%%%%%%%%%%%%%%%%%%%%%%%%%%%%%%%%%%%%%%%%%%%%
% The following lines define \boxaround, used to draw a box on the page.
% The parameter is the entire text of the resume. Must fit on one page!
%
% \boxaroundhmargin is the left & right margin around the text inside the box.
% \boxaroundvmargin is the top & bottom margin around the text inside the box.
% \boxrulethickness controls thickness of line used to draw the box.
% You can change these 3 things in the lines below:
%%%%%%%%%%%%%%%%%%%%%%%%%%%%%%%%%%%%%%%%%%%%%%%%%%%%%%%%%%%%%%%%%%%%%%%%%%%%%
\newdimen\boxrulethickness\newdimen\boxaroundhmargin\newdimen\boxaroundvmargin
\boxrulethickness=.5pt        %controls thickness of line 
\boxaroundhmargin=35pt        % about a half inch
\boxaroundvmargin=40pt        % to fit more text on page, make this smaller
%%%%%%%%%%%%%%%%%%%%%%%%% Don't read this stuff %%%%%%%%%%%%%%%%%%%%%%%%%%%%%%
\hsize=7.5in \vsize=10.5in             % use bigger dimensions for box
\newbox\MACboxA  \newdimen\MACdimenA
% \borderandboxit is used inside \boxaround:
\def\borderandboxit#1#2#3{\vbox{\hrule height#2\hbox{\vrule width#2\hskip#1\hskip-#2%
  \vbox{\vskip#1\relax#3\vskip#1}\hskip#1\hskip-#2\vrule width#2}\hrule height#2}}
%
\long\def\boxaround#1{\vskip6pt
  {\MACdimenA=\hsize \advance\MACdimenA by-\boxaroundhmargin
   \advance\MACdimenA by-\boxaroundhmargin   % once for each side
   \setbox\MACboxA=\hbox to \hsize{\hskip\boxaroundhmargin%\hss
                     \vbox{\hsize=\MACdimenA
                           \vskip\boxaroundvmargin #1
                           \vskip\boxaroundvmargin}\hss}%
   \borderandboxit{0pt}\boxrulethickness{\box\MACboxA}}%
  \vskip2pt plus0pt minus0pt
}
%%%%%%%%%%%%%%%%%%%  End of \boxaround macro %%%%%%%%%%%%%%%%%%%%%%%%%%%%%%%%%
 
\boxaround{ % put the text on the page inside a box  

  \name{\Large Paul PLANCHON\\[35pt]}
\address{\bf Adresse physique\\5 allée des platanes\\Appartement 414\\Cergy, 95000, France} 
\address{\bf Adresse numérique\\\texttt{planchon.io}\\
         paul@planchon.io\\ 06 73 34 16 44}
 
\begin{resume}

  \vskip 1cm
  
  \section{\textbf{Education}}
  \begin{itemize}
  \item Lycée Saint Joseph de Tivoli, voie scientifique 2005-2017
  \item EISTI génie mathématiques, Cergy, 2017 - aujourd'hui
  \end{itemize}

  \vskip 0.7cm
  
  \section{\textbf{Expériences professionelles}}
  \begin{itemize}
  \item 2015 - Enseignement de l'informatique et du cinéma à des jeunes de 8 ans à 15 ans à Jeunes Sciences Bordeaux
  \item 2016 - Réalisation de court métrage dans le but de promouvoir une méthode inovante d'élevage des vins
  \item 2018 - Création, traitement et analyse de base de donnée pour créer un carnet d'adresse de vignerons
  \item 2019 - Imagination d'un algorithme de résolution un problème d'optimisation sous contrainte dans le cadre du projet Culturation
  \end{itemize}
\end{resume}

  \vskip 0.7cm

\section{\textbf{Projets}}
\begin{itemize}
\item 2017 - Outil facilitant l'utilisation de \LaTeX en \texttt{python} (voir \texttt{ezTex})
\item 2018 - Moteur de jeu 3D (openGL) en \texttt{C++}
\item 2018 - Implémentation d'un algorithme génétique pour la résolution de Mastermind en \texttt{OCaml}
\item 2019 - Développement d'une librairie de deep-learning en \texttt{python}
\item 2019 - Serveur web et \texttt{API REST} en \texttt{C++} avec utilisation de \texttt{sockets}
\item 2019 - Création d'un librairie de création de vidéos (basé sur \LaTeX et \texttt{FFMPEG}) en \texttt{python}
\end{itemize}

  \vskip 0.7cm

\section{\textbf{Aptitudes}}
\begin{itemize}
\item Courant en français et de l'anglais à l'écrit comme à l'oral
\item Maitrise de \texttt{python, c++ et OCaml}
\item Maitrise de \LaTeX
\end{itemize}

\vskip 0.7cm

\section{\textbf{Activités}}
\begin{itemize}
\item Investisement dans une association d'aide à la découverte scientifique (Jeunes Science Bordeaux)
\item Pratique du Taekwondo \& Badminton
\item Niveau moniteur en hobbie 16 (voile)
\item Amour du voyage et du cinéma
\end{itemize}

\vfill} %    end the material being boxed.
\end{document}


